%
% 
%
\chapter{Introdução} \label{chap:introduction}

\emph{Isto é uma sinopse de capítulo. A ABNT não traz nenhuma normatização a
respeito desse tipo de resumo, que é mais comum em romances e livros técnicos.
}


A operação de separação de componentes de uma mistura por destilação tem uma longa história. Credita-se a sua
descoberta aos chineses durante a dinastia de Chou (221 a 1122 d.C.). Desde então, a produção de bebidas destiladas,
as chamadas ``bebidas dos deuses'', evoluíram juntamente com o progresso da civilização. Primeiro na Índia, depois na
Arábia e posteriormente na Inglaterra, em torno de 500 d.C., com a produção de ``mead'', uma bebida alcóolica
resultante da fermentação de mel e água. Mas surpreendentemente, apenas um milênio depois, por volta de 1500 d.C.,
tem-se notícia do primeiro uísque destilado na Escócia \apud{Forbes:1948}{Lockett:1986}. Embora a produção de bebidas
alcóolicas tenha uma grande importância no descobrimento e desenvolvimento da operação de destilação, nos dias de hoje,
a sua utilização é muito mais difundida nas indústrias químicas e de petróleo, apesar do grande consumo de energia
requirido \cite{Lockett:1986}.

% \section{Revisão Bibliográfica} \label{sec:revbiblio}

A simulação de processos de separação, em especial de colunas de destilação, é uma área muito importante na simulação
de processos dinâmicos. A destilação é uma das operações que mais demanda
energia dentre os processos químicos. Por exemplo, a destilação foi
responsável por 11\% do gasto de energia em indústrias dos EUA no ano de 1991,
conforme \apudonline{Humphrey:1992}{Koeijer:2004}.

Este trabalho tem como objetivo o desenvolvimento de um modelo rigoroso, dinâmico e de fácil aplicação nos mais
variados tipos de estudos, desde simples simulações de operação até otimizações e previsões de comportamentos
de parada e partida de plantas. Ou seja, um modelo projetado para situações onde uma representação fiel do
comportamento dinâmico é necessária.

Para a realização de tal objetivo será utilizado o simulador genérico de processos EMSO \cite{Soares:2003} e seu
ambiente de desenvolvimento de modelos. Os modelos gerados neste estudo fazem parte da biblioteca EML (EMSO Model
Library). A EML é distribuída no conceito de \emph{software} livre, disponibilizando todos os modelos via internet
e sem custo.

A pesquisa acerca da modelagem de colunas de destilação não é mais uma novidade nos dias de hoje. A simulação de
processos de separação por estágios de equilíbrio data de 1893, quando Sorel publicou equações simplificadas para
destilação estacionária. O equacionamento incluía balanços de massa, total e parcial, e balanços de energia. Porém,
as equações de Sorel não tiveram muita aplicação até o ano de 1921 quando serviram de base para o método gráfico
de cálculo de separação para sistemas binários desenvolvido por Ponchon e Savarit. Em 1925, com algumas
simplificações adicionais que eliminavam o balanço de energia, surgiu outro método gráfico, o método de
McCabe e Thiele \cite{Kooijman:1995a}. Seu maior valor é didático e não prático.
Este método utiliza gráficos construídos com relações de equilíbrio
entre os componentes a serem destilados, para determinar o número de estágios necessários na separação e a
razão de refluxo apropriada. As simplificações deste método são enormes, sendo projetado apenas para a separação
de misturas binárias \cite{Brooks:1993}.
Além disso, as vazões molares internas de líquido e vapor da coluna são consideradas constantes ao longo de cada seção, o
que significa que a a entalpia de condensação do vapor que chega ao prato deve
ser igual à entalpia de vaporização do líquido.

Os primeiros trabalhos com metodologias propostas para a solução de sistemas de separação modelados prato a prato
surgiram na década de 30. Só a partir da década de 50, com o advento do computador digital, foram realizados
investimentos sólidos no desenvolvimento de novos algoritmos e simuladores. Apesar deste investimento, apenas
modelos simplificados eram utilizados nas simulações devido à baixa capacidade
de processamento.

A partir da década de 70, os primeiros simuladores comerciais começaram a ser introduzidos na indústria e
o desenvolvimento de modelos rigorosos não parou mais.

Uma importante série de trabalhos encontrada na literatura foi desenvolvida por pesquisadores da Dinamarca,
Austrália e Argentina e contemplou todos os tópicos do desenvolvimento de modelos: o equacionamento, a solução
do sistema de equações resultante e a aplicação do modelo proposto.
No primeiro trabalho, \citeonline{Gani:1986} apresentam um modelo dinâmico genérico e rigoroso que considera
equilíbrio termodinâmico entre as fases.
O comportamento hidrodinâmico também foi modelado tornando possível o acompanhamento das vazões internas de
líquido e vapor da coluna e de eventos importantes como inundação e secagem dos pratos.
Para comprovar a eficiência e generalidade do modelo, vários exemplos de unidades industriais foram simulados,
validando o modelo com dados de planta.
Este trabalho foi muito importante e desencadeou o desenvolvimento de muitos outros.

Na sequência, \citeonline{Cameron:1986}, utilizando o modelo do trabalho anterior, abordaram aspectos numéricos
para a resolução do sistema de equações obtido.
Como naquela época não haviam códigos integradores de sistemas
algébrico-diferenciais (\emph{Dif-ferential-Algebraic Equations} - DAE), a simulação dinâmica de colunas de destilação requeria uma separação do sistema.
O conjunto de equações proveniente da modelagem era encarado como a união de um sub-sistema de equações
diferenciais ordinárias (\emph{Ordinary Differential Equations} - ODE) acoplado a outro sub-sistema de equações
algébricas. Assim, a solução do problema completo era obtida com a resolução sequencial dos dois blocos para cada
passo de integração.
No presente trabalho este problema é contornado, já
que o ambiente no qual os modelos foram desenvolvidos, o simulador \emso\
\cite{Soares:2003}, conta com integradores de sistemas DAE bem consolidados.

Aplicando os resultados obtidos nos dois trabalhos anteriores, \citeonline{Ruiz:1988} propuseram o desenvolvimento
de políticas de partida de colunas com base em simulações dinâmicas rigorosas.
Como resultado deste estudo foi possível caracterizar a partida de uma planta pela definição de três estágios distintos.
O primeiro, chamado \emph{estágio descontínuo}, é caracterizado pela descontinuidade das variáveis
(principalmente associadas à hidráulica) e pelo pequeno período de tempo de duração.
O segundo, chamado \emph{estágio semi-contínuo}, é identificado pelo comportamento transiente não-linear, embora não
haja mais descontinuidades, e pela aproximação das variáveis hidráulicas aos seus valores estacionários.
Já no terceiro, chamado \emph{estágio contínuo}, todas as variáveis alcançam comportamentos transientes lineares e
no final deste estágio o estado estacionário é atingido.

A aplicação dos modelos dinâmicos na predição do comportamento do sistema em partidas é muito importante, tanto para
a prevenção de possíveis problemas de controle e operabilidade, como para evitar desperdícios de matéria-prima ou
a geração de produtos fora de especificação.

Em seguida, \citeonline{Gani:1989} propuseram modificações no modelo dinâmico para o uso de um novo método
de resolução de simulações estacionárias. Este método tinha como objetivo alcançar maior robustez na solução de
problemas estacionários, visto que é mais difícil a obtenção de resultados com modelos puramente algébricos do
que com sistemas diferenciais. A troca entre o modo estacionário e o modo dinâmico dependia do cálculo dos resíduos
das equações. Se, com as estimativas iniciais, o problema algébrico não conseguia ser resolvido, o problema passava
para o modo dinâmico. No modo dinâmico, os estados eram estimados pela integração com as estimativas iniciais
correspondendo às condições iniciais. Assim que alguns critérios de integração eram alcançados, o problema voltava
para o modo estacionário que era então resolvido sem maiores dificuldades.

Os trabalhos desenvolvidos pelo grupo de \citeauthoronline{Gani:1986}, embora baseados em modelos rigorosos, com
várias correlações de hidrodinâmica tornando o modelo capaz de predizer com detalhes o comportamento hidráulico
das fases líquida e vapor na coluna, consideram o equilíbrio termodinâmico
entre as fases.
Mas, um grau ainda maior de detalhamento também é estudado: a condição de não-equilíbrio termodinâmico entre as fases.
Com a consideração de não-equilíbrio,
a resistência à transferência de calor e massa tanto inter como intrafase são
consideradas, gerando um sistema de equações mais complexo. No trabalho de
\citeonline{Biardi:1989}, a difusão transiente dos componentes é modelada pelas equações de Maxwell-Stefan.
A importância deste estudo está na comparação realizada entre modelos \emph{ideais} (com condição de equilíbrio
termodinâmico) e \emph{reais} (onde o equilíbrio é considerado apenas na interface líquido-vapor). Esta comparação
foi feita com dados experimentais de duas colunas em escala laboratorial e uma coluna industrial.
Os modelos reais apresentaram resultados mais precisos, mas em contrapartida requerem uma série de parâmetros
adicionais relacionados com a transferência de calor e massa entre as fases. Tais parâmetros são de difícil
obtenção e têm um alto grau de incerteza associado. Além disso, este tipo de modelo exige um maior custo
computacional em virtude da maior complexidade do sistema de equações gerado. Este tipo de equacionamento
é muito utilizado quando se quer reproduzir processos de destilação reativa.

Com o advento de computadores mais eficientes e integradores mais robustos, conseguiu-se resolver de forma mais
rápida os modelos desenvolvidos, inclusive os mais rigorosos.
Com a simulação ocorrendo em tempo real ou até algumas vezes mais rápido que isso, tornou-se possível a sua
utilização no treinamento de operadores e engenheiros. Porém, segundo \citeonline{Solomon:2007}, é pouco difundido
pois gasta-se mais tempo configurando a simulação que no treinamento propriamente dito.
No trabalho de \citeonline{Olsen:1997}, uma unidade de purificação de metano contendo três colunas de destilação
foi modelada, baseada no modelo desenvolvido por \citeonline{Gani:1986}.
O sistema de equações resultante apresentava 266.000 variáveis e 97.000 parâmetros e era simulado em uma velocidade
duas vezes mais rápida que a do processo.
Este simulador era utilizado principalmente para o treinamento de partidas e paradas das unidades, onde a predição
das características dinâmicas é de suma importância.


Além de simuladores de treinamento, muitas outras aplicações necessitam de modelos que predigam com uma boa fidelidade
o comportamento dinâmico de colunas de destilação. Entre as principais aplicações estão: simulações de partidas e
paradas, otimizações estacionárias e dinâmicas, estimações de parâmetros e implementação de sistemas de controle.

Os modelos disponíveis nos principais simuladores comerciais são geralmente do tipo ``caixa-preta'', impedindo que
o usuário possa realizar adaptações, aprimoramentos ou até mesmo simplificações no modelo, se achar necessário. No
caso em que não são fechados, os modelos disponíveis são construídos em linguagens de programação
de difícil compreensão, limitando o seu manuseio.
No simulador EMSO, todos os modelos são implementados em uma linguagem de
modelagem simples e podem ser inspecionados e editados livremente pelo usuário.

\nomenclature[Z]{EMSO}{Environment for Modeling, Simulation and Optimization}

Esta dissertação está dividida em seis capítulos, estruturados da seguinte forma:

O \autoref{chap:introduction} (este capítulo) trata da introdução e do objetivo deste trabalho.

O \autoref{chap:revisaobibliografica} apresenta as principais variações existentes na modelagem de colunas de destilação
encontradas na literatura, assim como suas vantagens e desvantagens frente às demais.

No \autoref{chap:moddesenvolvidos}, são apresentados os modelos desenvolvidos com suas considerações,
equacionamento e variações, bem como o ambiente utilizado para o desenvolvimento dos modelos,
o simulador EMSO.

Finalmente, no \autoref{chap:conclusoes}, as conclusões do trabalho são apresentadas bem como perspectivas de
futuros trabalhos a serem desenvolvidos.